
\begin{figure}[h]
    \vspace{0.25cm}
    \includegraphics[width=\textwidth]{%
      ../figures/4_network_models/4_network_models.pdf} %?? Remove!
  \vspace{-0.1cm}
  \caption{{\bf Overview of network models}. \textbf{A}--\textbf{D}:
    Graphic representation of the different network types. For each
    model the full network area is shown and the $N=1000$ neuron
    locations are indicated as gray dots. For a single cell (shown as
    $\bigtriangleup$, node number indicated in bottom right corner)
    target nodes of its outgoing connections are marked in color,
    revealing the typical connectivity in the model. %
    % 
    \textbf{E}--\textbf{F}: Visualization of the rewiring algorithm
    for a single edge between two neurons (source node $v$ shown as
    $\bigtriangleup$, target node $t(e)$ hatched in blue) at distance
    $x$. First, possible targets within distance $x \pm \varepsilon/2$
    are identified (nodes hatched in black). Then, from the pool of
    possible targets one node is chosen at random as the new
    target. The rewired edge projects from the original source vertex
    $v$ to the new target $t'(e)$ (hatched node in green). \textbf{G}:
    Connection probability for a random pair of nodes depends on their
    inter-node distance. In anisotropic networks (blue) and their
    rewired versions (green), the distance-depedent connectivity
    matches. The profile of anisotropic networks was used to generate
    the distance-dependent networks
    (purple). \textbf{H}:~Distance-dependent connection probability
    profile of the tuned network (red), matched to the findings of
    Perin et al.~\cite{Perin2011} (grey).}
\label{fig:4_net_models}
\end{figure}
