

\begin{figure}[h]
  \includegraphics[width=\textwidth]{%
    ../figures/two_neuron_connections/two_neuron_figure.pdf} %?? Remove!
  \caption{Overrepresentation of reciprocal connections %
    %
    \textbf{A}: In anisotropic (blue) and tuned anisotropic networks
    (red) occurrences of unconnected, unidirectionally and
    bidirectionally pairs were counted and these values were divided
    by an expected count of $N(N-1)(1-p)^2$ for unconnected,
    $N(N-1)p(1-p)$ for unidirectionally connected and $N(N-1)p^2$ for
    reciprocally connected pairs. Here $p=0.116$ is connection density
    in all network types. Reciprocally connected neuron pairs are
    overrepresented in both anisotropic and tuned anisotropic
    networks. %
    %
    \textbf{B}: Overrepresented reciprocal pairs also occur in rewired
    and and distance-dependent networks. For unconnected (left),
    unidirectionally connected (middle) and bidirectionally connected
    pairs (right), the relative count. While general
    overrepresentation is found, no significant difference to the
    rewired mode is identified %
    %
    \textbf{C:} Distance-dependent connection probability in tuned
    anisotropic networks (red) is matching the connection probabilities
    of layer 5 pyramidal cell in rat somatosensory cortex (grey) as
    found by \textcite{Perin2011}. %
    %
    \textbf{D:} Distance-dependent probabilities for reciprocal
    connections is much lower in tuned anisotropic (red circles) than
    in experimental data (grey) and matches the probabilities expected
    from C under the assumption of independent realization of
    connections (red dashed line).}
\label{fig:two_neuron}
\end{figure}


