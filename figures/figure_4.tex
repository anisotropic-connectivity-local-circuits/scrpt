\begin{figure}[h]
  \medskip
  \centering
  \includegraphics[width=0.95\textwidth]{%
    % /home/fh/sci/lab/aniso_netw/ploscb_18/fig/main/fig4_three_motifs.pdf}
    /home/fh/sci/rsc/aniso_netw/pub/arxiv18/figures/three_motifs/three_motifs_3x1.pdf}
  \caption{Anisotropy in connectivity induces overrepresentation of
    specific three neuron motifs. \textbf{A}:~The ratio between
    observed triplet motif counts and counts expected from neuron pair
    connectivity is shown for anisotropic (blue), rewired (green) and
    distance-dependent networks (purple). Of each network type 5
    network instances were created (see Methods for parameters). From
    each network we sampled $n=300000$ random groups of three neurons
    and recorded which of the motifs 1-16 they belong to. The counts
    for each motif was then divided by the number of occurrences
    expected from neuron pair probabilities (see
    \ref{SI-sec:3motif_stat} for details). Errorbars indicate
    SEM. Motifs labeled with * were reported by \textcite{Perin2011}
    to be overrepresented in layer 5 pyramidal cell of rat
    somatosensory cortex. \textbf{B}: Motif count overrepresentation
    as in A, here for tuned anisotropic networks (red) and their
    rewired (grey) and distance-dependent versions (purple).}
  \label{fig:3_motifs}
\end{figure}
%