% Please keep the abstract below 300 words
\section*{Abstract}
%
Nonrandom connectivity patterns have been repeatedly found in local cortical
networks, but it remains unclear which connection principles underlie this form of network organization. Here we present a simple geometric random network model that implements connectivity as one would expect from neuron-typical axon and dendrite morphology. The statistics of are well reproduce in our model. We are able to show that the emerging nonrandomness is indeed due to networks anisotropy in connectivity. Our results show that characteristic neuron morphology must therefore be considered as an important aspect of the connection principles in the cortex.



Suggesting that features of the pyramdial neuron's
stereotypical morphology can be the cause for such non-randomness, a
simplistic geometric random network model is introduced reflecting
"anisotropy in neural connectivity", the observation that synapses of
cortical pyramidal cells tend to cluster around the main axon's
projection. Analysis of the network's connectivity reveals patterns
closely resembling the findings in cortical circuits. Characteristic
neuron morphology must therefore be considered as an important aspect
of the underlying connection principles in the cortex. Reflecting
network non-random network connectivity faithfully, the proposed model
offers itself for further investigation of the consequences of
patterns consequences in networks dynamics and learning.


