
We consider four types of random network models and analyze their
connectivity structure. The \textit{anisotropic} and \textit{tuned
anisotropic} network models implement the anisotropy in the
. The \textit{rewired} version of those networks serve as a baseline
. The \textit{distance-dependent} network allows for further
distinction between effects.

\subsection*{Anisotropic networks}

The anisotropic network model is the simplest form of a random network
featuring anisotropy in connectivity. In the model the $N$ nodes of
the network are first distributed uniformly at random on a square of
side length $L$. Each node $v$ is randomly assigned an angle
$\alpha \in [0,2\pi)$. Directed connections are then established from
$v$ to other nodes with distance less than $\frac{w}{2}$ to the
projection ray originating from $v$ with angle $\alpha$
(Fig.~\ref{fig:aniso_net}, \ref{fig:4_net_models}A). We chose
$N=1000$, leaving the ratio $\frac{w}{L}$ as a free parameter of the
model determining the connection density of the network. We selected
$L=\SI{296}{\micro\meter}$ and determined $w=\SI{74.6}%
{\micro\meter}$ to obtain a connection density of $p=11.6$ as in
\cite{Song2005}. For a formal mathematical definition of the
anisotropic network model see supplementary material
(\ref{SI-sec:aniso_model}).



\begin{figure}[h]
  \vspace{0.25cm}
  \centering
  \includegraphics[width=0.96\textwidth]{%
    ../figures/aniso_network_model/aniso_network_model.pdf} %?? Remove!
  \vspace{0.1cm}
\caption{{\bf Anisotropic network model}. \textbf{A}: Algorithm for the
  generation of the network. First the $N$ nodes (grey dots) are
  distributed randomly on the square surface. Then, for each source
  node (shown as $\bigtriangleup$, identification number in the bottom left
  corner) a random angle $\alpha \in [0,2\pi)$ is chosen (1.). The
  target nodes for $\bigtriangleup$ are found within euclidean distance of
  less than $\frac{w}{2}$ from the projection originating at $\bigtriangleup$
  in the direction of $\alpha$ (2., projection shown in blue,
  dashed). Connections are then established from the source to the
  target nodes (3., targets of $\bigtriangleup$ shown in blue, remaining
  unconnected nodes in grey). \textbf{B}: Targets of another node
  (identification number 54) with different location and projection
  angle $\alpha$.}
\label{fig:aniso_net}
\end{figure}




\bigskip

\subsection*{Rewired networks}
%
In order to identify connectivity features in a network that are a
direct result of its anisotropy in spatial connectivity, we introduce
rewired versions of anisotropic networks. In the rewiring process,
existing edges are randomly assigned new targets in a way so that
anisotropy is eliminated while other important connectivity statistics
such as distance-dependent connectivity and in- and out-degree
distributions are preserved.
% (Fig.~\ref{fig:4_net_models}B)


a modified
network that lacks anisotropy but is otherwise equivalent to the
original graph is required. This is achieved for a rewiring algorithm


we introduced a rewiring algorithm that eliminates anisotropy but preserves other important 


ia network without anisotropy as close as possible to. For this we introduce a rewiring algorithm operating
on the anisotropic networks.

Thus, to
discern features that are truly 

it is necessary to have a reference network that doesn't
feature anisotropy, but is otherwise equivalent to the anisotropic
model.

For this we introduced a rewiring algorithm on networks that
preserves first-order connectivity statistics %??  as well as the
distance-dependent connection probability profile. In this process
each connection with a spatial distance $x$ between source and target,
is rewired to a new random target that differs not more than
$\varepsilon$ from $x$ in spatial distance to the source vertex
(Fig.~\ref{fig_model} E-F). The resulting network has greatly reduced
anisotropy in connectivity, resembling that of a distance-dependent
network (supporting information).

Mention the outcome of the rewiring. Some very edges are lost in the process! %??

By rewiring only a section of all connections, a \textit{partial
  rewiring} produces a state that retains some anisotropy...



The relative width of the rewiring parameter was chosen as $\varepsilon / E = 0.05$ (??) to achieve an effective rewiring while closely maintaining the distance-dependent connection probability distribution of the original networks (see SI ??).




\begin{figure}[h]
    \vspace{0.25cm}
  \includegraphics[width=\textwidth]{../figures/4_network_models/4_network_models.pdf} %?? Remove!
  \vspace{0.15cm}
\caption{{\bf Network models} \textbf{A}--\textbf{D}. Graphic representation of the different network types. For each model the full network area is shown and the $N=1000$ neuron locations are indicated as gray dots. For a single cell (triangle, node number shown in bottom right corner) target nodes of its outgoing connections are marked in color, revealing the typical connectivity in the model. \textbf{E}--\textbf{F}. Visualization of the rewiring algorithm for a single edge between two neurons of distance $x$. First, possible targets within distance $x \pm \varepsilon/2$ are identified (hatched nodes). Then, from the possible targets one node is chose at random and the rewired edge projects from the original source vertex $v$ to the new target $t'(e)$. \textbf{G} Connection probability between two nodes dependent on their distance matches in the networks from A,B and C. \textbf{H} Distance-dependent connection probability profile of the tuned network, matched to the findings of Perin et al.~(2011) \cite{Perin2011}.
}
\label{fig:4_net_models}
\end{figure}


\subsection*{Distance-dependent networks}

A further reference network model are distance-dependent networks, in which each possible connection is realized independently with a probability $p(x)$, that depends only on the Euclidean distance $x$ between the source and target node (Fig.~\ref{fig:4_net_models}C). We refer to $p(x)$ as the distance-dependent connectivity profile.


\subsection*{Tuned anisotropic networks}




