

We consider four types of random network models and analyze their
connectivity structure. The \textit{anisotropic} and \textit{tuned
anisotropic} network models implement the anisotropy in the
. The \textit{rewired} version of those networks serve as a baseline
. The \textit{distance-dependent} network allows for further
distinction between effects.

\subsection*{Anisotropic network model}

The anisotropic network model is the simplest form of an
implementation of anisotropy in connectivity in a network. The $N$
nodes of the network are distributed uniformly at random on a square
of side length $L$. Each node $v$ is randomly assigned an angle
$\alpha \in [0,2\pi)$. Directed connections are then established from
$v$ to other nodes with distance less than $\frac{w}{2}$ to the
projection ray originating from $v$ with angle $\alpha$
(Fig.~\ref{fig_model}A). The ratio of the width of $w$ to side length
$L$ is a free parameter of the model . We here choose
$L=\SI{296}{\micro\meter}$ and $w=$

The model is implemented as an adapted version of a two-dimensional
directed random geometric graph: additionally to the random position
on the unit square, each node is assigned a uniformly distributed axon
projection angle $a$. For each vertex $v$, directed connections are
then established to nodes with distance $\leq \frac{w}{2}$ to the
projection ray originating from $v$ with angle $a$
(Fig.~\ref{fig_model} A). This \textit{anisotropic network
  model} is determined by two parameters; network size $N$ and the
\enquote{axon width} $w$.




\subsection*{Rewired networks}
%
In order to understand which structural features are shaped by the
anisotropy in the network, it is necessary to have a reference network
that doesn't feature anisotropy, but is otherwise equivalent to the
anisotropic model. For this we introduced a rewiring algorithm on
networks that preserves first-order connectivity statistics %??
as well as the distance-dependent connection probability profile. In
this process each connection with a spatial distance $x$ between
source and target, is rewired to a new random target that differs not
more than $\varepsilon$ from $x$ in spatial distance to the source
vertex (Fig.~\ref{fig_model} E-F). The resulting network has greatly
reduced anisotropy in connectivity, resembling that of a
distance-dependent network (supporting information).

Mention the outcome of the rewiring. Some very edges are lost in the process! %??

By rewiring only a section of all connections, a \textit{partial
  rewiring} produces a state that retains some anisotropy...



The relative width of the rewiring parameter was chosen as $\varepsilon / E = 0.05$ (??) to achieve an effective rewiring while closely maintaining the distance-dependent connection probability distribution of the original networks (see SI ??).



\subsection*{Distance-dependent networks}

To record in how far the rewired model resembles networks implemented
purely from a distance-dependent. In order to test for this a standard
distance-dependent was implemented as well.




\subsection*{Tuned networks}

Finally, 



\begin{figure}[h]
    \vspace{0.25cm}
  \includegraphics[width=\textwidth]{../figures/4_network_models/4_network_models.pdf} %?? Remove!
  \vspace{0.15cm}
\caption{{\bf Network models} \textbf{A}--\textbf{D}. Graphic representation of the different network types. For each model the full network area is shown and the $N=1000$ neuron locations are indicated as gray dots. For a single cell (triangle, node number shown in bottom right corner) target nodes of its outgoing connections are marked in color, revealing the typical connectivity in the model. \textbf{E}--\textbf{F}. Visualization of the rewiring algorithm for a single edge between two neurons of distance $x$. First, possible targets within distance $x \pm \varepsilon/2$ are identified (hatched nodes). Then, from the possible targets one node is chose at random and the rewired edge projects from the original source vertex $v$ to the new target $t'(e)$. \textbf{G} Connection probability between two nodes dependent on their distance matches in the networks from A,B and C. \textbf{H} Distance-dependent connection probability profile of the tuned network, matched to the findings of Perin et al.~(2011) \cite{Perin2011}.
}
\label{fig:4_net_models}
\end{figure}

