
We consider four types of random network models and analyze their
connectivity structure. The anisotropic and tuned anisotropic networks
are simple models that implement anisotropy in spatial connectivity as
motivated from steoreotypical pyramidal cell morphology. The rewired
versions of those networks lack anisotropy but can be seen as
otherwise equivalent in their connectivity and provide an important
reference for this study. The distance-dependent networks allow for
further distinction between effects on the network connectivity
structure that are direct or indirect consequence of anisotropy.

\subsection*{Anisotropic networks}

The anisotropic network model is the simplest form of a random network
featuring anisotropy in connectivity. In the model the $N$ nodes of
the network are first distributed uniformly at random on a square of
side length $E$. Each node $v$ is randomly assigned an angle
$\alpha \in [0,2\pi)$. Directed connections are then established from
$v$ to other nodes with distance less than $\frac{w}{2}$ to the
projection ray originating from $v$ with angle $\alpha$
(Fig.~\ref{fig:aniso_net}). We chose $N=1000$, leaving the ratio
$\frac{w}{E}$ as a free parameter of the model determining the
connection density of the network. We selected
$E=\SI{296}{\micro\meter}$ and determined $w=\SI{74.6}%
{\micro\meter}$ to obtain a connection density of $p=11.6$ as in
\cite{Song2005}. For a formal mathematical definition of the
anisotropic network model see supplementary material
(\ref{SI-sec:aniso_model}).



\begin{figure}[h]
  \vspace{0.25cm}
  \centering
  \includegraphics[width=0.96\textwidth]{%
    ../figures/aniso_network_model/aniso_network_model.pdf} %?? Remove!
  \vspace{0.1cm}
\caption{{\bf Anisotropic network model}. \textbf{A}: Algorithm for the
  generation of the network. First the $N$ nodes (grey dots) are
  distributed randomly on the square surface. Then, for each source
  node (shown as $\bigtriangleup$, identification number in the bottom left
  corner) a random angle $\alpha \in [0,2\pi)$ is chosen (1.). The
  target nodes for $\bigtriangleup$ are found within euclidean distance of
  less than $\frac{w}{2}$ from the projection originating at $\bigtriangleup$
  in the direction of $\alpha$ (2., projection shown in blue,
  dashed). Connections are then established from the source to the
  target nodes (3., targets of $\bigtriangleup$ shown in blue, remaining
  unconnected nodes in grey). \textbf{B}: Targets of another node
  (identification number 54) with different location and projection
  angle $\alpha$.}
\label{fig:aniso_net}
\end{figure}




\bigskip

\subsection*{Rewired networks}
%
In order to identify connectivity features that are a direct result of
a network's anisotropy in spatial connectivity, we introduce rewired
versions of anisotropic networks. In the rewiring process existing
edges are randomly assigned new targets so that anisotropy is
eliminated while other important connectivity statistics, such as
distance-dependent connectivity and in- and out-degree distributions,
are preserved. For example, Fig.~\ref{fig:4_net_models}A shows the
targets of a node in the original anisotropic network which are to be
rewired. The same node is then shown in Fig.~\ref{fig:4_net_models}B
with its new targets in the rewired network.
%
% We simply call the modified networks rewired networks (Fig.~\ref{fig:4_net_models}B).

The rewiring algorithm has two parameters -- the rewiring fraction
$\eta$ of edges to be rewired and the rewiring margin $\varepsilon$
that determines how many rewiring targets are on average available. In
the process any given edge of the graph gets rewired with probability
$\eta$. If an edge $e$ with source node $s(e)$ and target node $t(e)$
is to be rewired, at first the Euclidean distance $x$ between $s(e)$
and $t(e)$ is recorded. Potential new targets for $e$ are then all
nodes that have a distance to $s(e)$ larger than
$x-\frac{\varepsilon}{2}$ but smaller than $x+\frac{\varepsilon}{2}$
(Fig.~\ref{fig:4_net_models}E). From the pool of potential new targets
are chosen at random until a target is found that does not already
receive input from $s(e)$ either through a previously rewired edge or
an edge that will not be rewired if $\eta <1$. The original edge $e$
is then replaced by a new edge $e'$ from $s(e)$ to the new target
(Fig.~\ref{fig:4_net_models}F). If no such target is available the
edge is discarded and will not appear in the rewired network.

We chose a rewiring margin $\varepsilon / E = 0.05$ to provide a large
enough average number of rewiring targets while closely maintaining
the distance-depedent connecitivty in the network (see
Section~\ref{SI-sec:rewiring_margin}). A rewired network, if not
otherwise specified, refers to a fully rewired version ($\eta=1$) of
the current network. 



\begin{figure}[h]
    \vspace{0.25cm}
  \includegraphics[width=\textwidth]{../figures/4_network_models/4_network_models.pdf} %?? Remove!
  \vspace{0.15cm}
\caption{{\bf Network models} \textbf{A}--\textbf{D}. Graphic representation of the different network types. For each model the full network area is shown and the $N=1000$ neuron locations are indicated as gray dots. For a single cell (triangle, node number shown in bottom right corner) target nodes of its outgoing connections are marked in color, revealing the typical connectivity in the model. \textbf{E}--\textbf{F}. Visualization of the rewiring algorithm for a single edge between two neurons of distance $x$. First, possible targets within distance $x \pm \varepsilon/2$ are identified (hatched nodes). Then, from the possible targets one node is chose at random and the rewired edge projects from the original source vertex $v$ to the new target $t'(e)$. \textbf{G} Connection probability between two nodes dependent on their distance matches in the networks from A,B and C. \textbf{H} Distance-dependent connection probability profile of the tuned network, matched to the findings of Perin et al.~(2011) \cite{Perin2011}.
}
\label{fig:4_net_models}
\end{figure}




\subsection*{Distance-dependent networks}

A further reference network model are distance-dependent
networks. Here, each possible connection is realized independently
with a probability $p(x)$, that depends only on the Euclidean distance
$x$ between the source and target node
(Fig.~\ref{fig:4_net_models}C). We refer to $p(x)$ as the
distance-dependent connectivity profile. 



\subsection*{Tuned anisotropic networks}

Distance-depedent connectivity in anisotropic networks is highly
specific (Fig.~\ref{fig:4_net_models}G) and likely not a realistic
model for biological systems. In order to obtain distance profiles
more closely resembling neural circuit connectivity, we introduce a
variation of the anisotropic networks in which $w$ is not fixed, but
depends on the distance $x$ from the source node
(Fig.~\ref{fig:4_net_models}D). Thus the width $w$ thus becomes a
distance-dependent function $w(x)$ in tuned anisotropic networks. We
determined $w(x)$ to obtain a distance-dependent connection
probability as found by \textcite{Perin2011}
(Fig.~\ref{fig:4_net_models}H, see Section~\ref{SI-sec:tuned_networks}
for details). In this case the surface side length
$E=\SI{296}{\micro\meter}$ was picked such that the connection density
matched $p=0.116$ as in \textcite{Song2005} and the other network types.







