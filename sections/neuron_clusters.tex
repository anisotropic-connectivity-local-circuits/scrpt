
\subsection*{Neuron clusters}


\begin{figure}[h!]
  \centering \includegraphics[width=0.95\textwidth]{%
    /home/fh/sci/rsc/aniso_netw/pub/arxiv18/figures/neuron_clustering/nc7_3c-3f_6c-6f_8c-8f_12c.pdf}
  \caption{Densely connected groups of neurons appear more frequently
    in networks with anisotropy. \textbf{A}: In anisotropic (blue) and
    tuned anisotropic networks (red) we sampled $n=2.5 \times 10^6$
    groups of 3 neurons and counted the connections within each
    group. This process was then repeated for the rewired anisotropic
    and tuned anisotropic networks. For each number of connections, we
    plotted the difference between occurrences of this connection
    count in random groups in anisotropic and rewired networks,
    divided by the number of occurrences in rewired networks. Fully
    connected groups did not occur in tuned anisotropic
    networks. \textbf{B}: Relative frequency of number of connections
    occurring in groups of 3 neurons in tuned anisotropic (red) and
    rewired tuned anisotropic network (black). \textbf{C-D}: As in A-B
    for groups of 6 neurons. \textbf{E-F}: As in A-B for groups on 8
    neurons. \textbf{G}: As in A for groups of 12 neurons.}
\label{fig:neuron_clusters}
\end{figure}




Specifically, the high edge counts in neuron clusters \textit{does
  not} depend on how anisotropy was implemented (tuned or normal)!

While rewiring, some connections have been lost (numbers!), that means
some of the overrepresentation might be due to that as this
relationship is strict (no new connections get introduced). It was
shown that small deviations *can* have a significant effect - see for
example 85c70a9f\_1v1. However, in average no such is noticeable as
expected. 

The significant overrepresentation from rewired to distance-dependent
is something important to think about. It kind of destroys the point
I'm trying to make: Anisotropy makes all the things! So, it's very
important to understand: Is it really true that the rewired-distance
effect is strong? (So far this has been only tested for one case:
8counts) Then, it's crucial to understand where this difference is
coming from. What aspect is in rewired, but not in distance, that
produces this?
