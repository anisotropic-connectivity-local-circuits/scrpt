
We generated samples of anisotropic and tuned anisotropic networks and
their respective rewired and distance-dependent versions (see
Methods). We analyzed the connectivity structures in these networks
and tested which connectivity features emerge from anisotropy in
spatial connectivity.

\subsection{Degree distributions}

The number of connections that a node in the network receives
(in-degree) or makes to other nodes (out-degree) is an important
indicator of the network structure and can strongly affect its
dynamics \cite{Roxin2011a, Martens2017}. In our model in-degree
distributions in the anisotropic, rewired and distance-dependent
networks matched, but were in general broader than the in-degree
distribution in random networks of the same connection probability
(\figref{fig:io_deg}A-B). Out-degree distributions on the other hand
were very broad in anisotropic and rewired networks while
distance-dependent networks maintained a distribution similar to the
in-degree (\figref{fig:io_deg}C-D). The stretched out distributions in
anisotropic networks are a result of some axons being \enquote{cut
  off} at a short length, while other axons travel across the network
surface in full length and make numerous connections along the
way. The observed difference between in-degree and out-degree
distributions in anisotropic networks could be expected in connectomes
inferred from slice experiments as well. For this study, we maintain
both the distance-dependent and rewired networks as reference models
such that features that appear in anisotropic networks but not in
rewired and distance-dependent networks can be truthfully attributed
to anisotropy in connectivity.



\begin{figure}[h]
  \includegraphics[width=\textwidth]{%
    ../figures/io_degrees/io_degrees_2x3.pdf} %?? Remove!
  \caption{Degree distributions and evaluation of rewiring algorithm %
    %
    \textbf{A}: In-degree distribution in anisotropic (blue), rewired
    anisotropic (green, dashed), distance-dependent (purple, dashed)
    and random networks (grey). In random networks, a connection between
    two neurons exists independently with probability $p$ and the in- and
    out-degree distributions are distributed according to a binomial
    distribution $B(N,p)$, where $N=999$ is the number of targets or
    possible sources for each node. %
    %
    \textbf{B}: In-degree distribution in the tuned anisotropic (red),
    rewired tuned anisotropic (grey, dashed), distance-dependent
    (purple) and random networks (grey). %
    %
    \textbf{C}: Out-degree distribution for networks as in A. %
    %
    \textbf{D}: Out-degree distribution for networks as in B. %
    %
    \textbf{E}: Probability density distribution of the anisotropy
    degree $\lambda$ in the networks, showing that there is strong
    anisotropy in connectivity in anisotropic networks (blue) but
    not in rewired (green) or distance-dependent networks (purple). %
    %
    \textbf{F}: As in E but for tuned anisotropic (red), rewired tuned
    anisotropic (grey) and distance-dependent networks (purple).}
\label{fig:io_deg}
\end{figure}




        
\subsection{Rewired networks}

Next we to verified that the rewiring algorithm indeed eliminates
anisotropy in the network. For this we measured the degree of
anisotropy in the targets of a neuron $v$ by summing the normalized
vectors pointing from $v$ to its connected target neurons. The
resulting vector's norm, divided by the number of targets of $v$, is
then a normalized measure $\lambda(v)$ for the anisotropy in spatial
connectivity of $v$ -- if target neurons are spread isotropically
around the neuron $\lambda(v)$ is close to $0$, if almost all targets
lie in the same direction from $v$ the value of $\lambda(v)$ is close
ton $1$ (for a formal definition of $\lambda$ see supplementary
material \ref{SI-sec:aniso_measure}).


Anisotropic and tuned anisotropic networks have distributions of
$\lambda$ with a peak close to $1$ (\figref{fig:io_deg}E-F). After
rewiring this peak completely vanishes and the distribution of
$\lambda$ in rewired networks approximately matches the distribution
in distance-dependent networks. In summary, after rewiring networks
have an essentially unchanged equal distance-dependent connectivity
(\figref{fig:4_net_models}G), maintain the in- and out-degree
distribution (\figref{fig:io_deg}A-D) but lack any anisotropy in
spatial connectivity that might have been present. With this rewired
networks are suitable to test which connectivity features found in
anisotropic networks are a direct result of its anisotropy in
connectivity.


% We verified that the rewiring alg For
% this Indeed, taking the vector (see Methods) that rewiring more and
% more of the graph shifts the. Distance-dependent connection
% probabilities (\figref{fig:4_net_models}G) and degree distributions
% (\figref{fig:io_deg}A-D) remain by architecture of the rewiring
% algorithm unchanged. We further tested if additionally rewiring.

% Taken together, rewired networks provide a reference to distinguish
% feature that arrise solely due to anisotropy in spatial connectivity.


% \begin{itemize}
% \item  degree distribution unchanged
% \item  anisotropy reduced (measure)
% \item  further rewiring (rewiring the rewired network) does not change things  
% \end{itemize}




