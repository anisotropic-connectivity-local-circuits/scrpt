
We generated samples of anisotropic and tuned anisotropic networks and
their respective rewired and distance-dependent versions (see
Methods). We analyzed the connectivity structures in these networks
and tested which connectivity features emerge from anisotropy in
spatial connectivity.

\subsection{Degree distributions}

The number of connections that a node in the network receives
(in-degree) or makes to other nodes (out-degree) is an important
measure of the network structure (citations??). In-degree
distributions in the original, rewired and distance-dependent networks
matched, but were in general broader than the in-degree distribution
in random networks of the same connection probability
(\figref{fig:io_deg}A,C). Due to axons extending



Taken together, rewired networks provide a reference to distinguish
feature that arrise solely due to anisotropy in spatial connectivity.



\begin{figure}[h]
  \includegraphics[width=\textwidth]{%
    ../figures/io_degrees/io_degrees_2x3.pdf} %?? Remove!
  \caption{Degree distributions and evaluation of rewiring algorithm %
    %
    \textbf{A}: In-degree distribution in anisotropic (blue), rewired
    anisotropic (green, dashed), distance-dependent (purple, dashed)
    and random networks (grey). In random networks, a connection between
    two neurons exists independently with probability $p$ and the in- and
    out-degree distributions are distributed according to a binomial
    distribution $B(N,p)$, where $N=999$ is the number of targets or
    possible sources for each node. %
    %
    \textbf{B}: In-degree distribution in the tuned anisotropic (red),
    rewired tuned anisotropic (grey, dashed), distance-dependent
    (purple) and random networks (grey). %
    %
    \textbf{C}: Out-degree distribution for networks as in A. %
    %
    \textbf{D}: Out-degree distribution for networks as in B. %
    %
    \textbf{E}: Probability density distribution of the anisotropy
    degree $\lambda$ in the networks, showing that there is strong
    anisotropy in connectivity in anisotropic networks (blue) but
    not in rewired (green) or distance-dependent networks (purple). %
    %
    \textbf{F}: As in E but for tuned anisotropic (red), rewired tuned
    anisotropic (grey) and distance-dependent networks (purple).}
\label{fig:io_deg}
\end{figure}




        
\subsection{Rewired networks}

We verified that the rewiring algorithm indeed works as
expected. For this Indeed, taking the vector (see Methods) that rewiring more
and more of the graph shifts the. Distance-dependent connection
probabilities (\figref{fig:4_net_models}G) and degree distributions
(\figref{fig:io_deg}A--D) remain by architecture of the rewiring
algorithm unchanged. We further tested if additionally rewiring.

% \begin{itemize}
% \item  degree distribution unchanged
% \item  anisotropy reduced (measure)
% \item  further rewiring (rewiring the rewired network) does not change things  
% \end{itemize}




