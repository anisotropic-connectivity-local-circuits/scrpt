\clearpage
\pagebreak
\newpage
\subsection*{Three neuron motifs}

Multiple paired whole-cell recordings in slices of rat visual and
somatosensory cortex revealed nonrandom structures in layer 5
pyramidal cells beyond pair connections -- the connection motifs
occurring in groups of three neurons deviate significantly of what one
would predict from the occurrence neuron pair
connections~\cite{Song2005, Perin2011}. Song et al.~for example found
the probability for a neuron pair $(u,v)$ to have a connection
$u \rightarrow v$ to be $p_{u \rightarrow v}= 0.0615$ and the
probability for a bidirectional connection was
$p_{u \leftrightarrow v} = 0.0542$. The probability for a group of
three randomly selected neurons to show the connection motif 14 as
labelled in \figref{fig:3_motifs}A, can then be computed from the pair
probabilities as
\[
  \textbf{P}(\text{motif 14})= 3 \,p_{u \rightarrow v}^2
                                 \, p_{u \leftrightarrow v},
\]
as the pattern is composed of two unidirectionally and one
bidirectionally connected pair and there are 3 possible ways to
arrange the pairs to obtain the motif (for details and combinatorics of
all patterns see supplementary material
\ref{SI-sec:3motif_stat}). Here we also analyzed the counts of triplet
motifs in the anisotropic networks and compared it with the expected
counts predicted from the pair statistics extracted before. We found a
highly specific distribution of three neuron motifs in anisotropic
networks (\figref{fig:3_motifs}A) and tuned anisotropic networks
(\figref{fig:3_motifs}B).%
%
\begin{figure}[h]
%\includegraphics[width=0.99\textwidth]{../img/song_motifs.png} %??
%%Remove!

%\newgeometry{left=0.75in,right=0.75in} % ?? s
\centering
% \advance\leftskip-2.25in %??
% \includegraphics[width=6.2in]{../img/song_motif_single_img_colors.png}

%% \includegraphics[width=\textwidth]{../img/song_motif_single_img_colors.png}
\includegraphics[width=\textwidth]{/home/fh/sci/lab/aniso_netw/ploscb_18/fig/main/fig4_three_motifs.png}

%\includegraphics[width=6.8in]{/users/hoffmann/Downloads/9f9452a2_plos.pdf}
% ?? %Remove!
%\restoregeometry %??
\caption{{\bf Over- and underrepresentation of three-neuron motifs due
  to anisotropy}
  Counting the occurrences of three-neuron motifs in anisotropic,
  rewired and distance-dependent networks, we find that }
\label{fig_three-neuron}%?? proper label
\end{figure}

%
% To expose the effect of anisotropy in connectivity on the statistics
% of the occurrence of motifs composed of three neurons, in each of the
% three graph types (anisotropic, rewired, distance-dependent) a sample
% of 3 graphs was taken and in each of those graphs the motifs of 300000
% random triplets of neurons were registered. % song_alt_dist_N3_300k 
% The observed occurrence of three-neuron motifs distinctly differed
% from what one might expect from relative frequencies of connections in
% neuron pairs: Assuming independence, the expected relative frequency
% of the 16 different motifs were calculated from probabilities $p_u,
% p_s$ and $p_r$ from above %?? are they really mentioned abov?.  
% \ref{SI_song_motif_comp}%??
% Displaying the relative counts (counts extracted from graphs divided
% by counts expected from neuron pair statistics) in
% Fig.~\ref{fig_three-neuron}, we find that anisotropy has significant
% impact of distribution of motifs. 

These data match well . Here, over- and underrepresentation of certain
motifs is a direct result of anisotropy in spatial connectivity as
rewired or distance-dependent versions of the networks do not show the
same degree of deviation from the expected motif occurrence.

The strongest deviations were
observed in motifs 8,10,12,14,15 and 16 (overrepresented) and motif 9 ??
(underrepresented). Rewired and distance-dependent networks show a
deviation from the expectation as well, however, the effect is much
stronger in anisotropic networks

Interestingly, this effect is much stronger than in
distance-dependent, where one might expect significant over- and
underrepresentations as well. Finding two connected pairs in a triplet
indicates a spatial closeness, that can be transferred to third pair
as well - thus one should find a higher connection chance. We find
that in comparable distance-dependent networks, only


