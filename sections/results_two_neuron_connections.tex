\subsection*{Two neuron connections}

A prominent structure of nonrandom connectivity in cortical circuits
is the abundance of reciprocal connections. In networks in which nodes
are connected randomly with probability $p$, bidirectionally connected
pairs occur with a low probability of $p^2$. Experimental results in
 rat visual cortex \cite{Song2005} and somatosensory cortex
\cite{Markram1997, Perin2011} however indicate that reciprocally
connected pairs of layer 5 pyramidal neurons occur much more often
than in a randomly connected network of the same connection density.

We analyzed whether such bidirectionally connected pairs also occur
abundantly in anisotropic networks. Indeed, we find that reciprocal
connections are overrepresented in both anisotropic in tuned
anisotropic networks (\figref{fig:two_neuron}A). However, is this
overrepresentation a direct result of anisotropy in spatial
connectivity? To test this we also determined the occurrence of
unconnected, unidirectionally and bidirectionally connected pairs in
rewired and distance-dependent networks (\figref{fig:two_neuron}B). We
found that the different pair motifs occur approximately equally often
in the three network types. Reciprocal connections are thus
overrepresented in anisotropic, rewired and distance-dependent
networks to equal degree.



\begin{figure}[h]
\includegraphics[width=\textwidth]{../figures/two_neuron_connections/two_neuron_figure.pdf} %?? Remove!
\caption{{\bf Overrepresentation of two neuron connections not is not
    induced by anisotropy}
A-B: While general overrepresentation is found, no significant
difference to the rewired mode is identified C-E: Overrepresentation
as found by Perin et al. is distance-independent and is not affected
by anisotropy}
\label{fig_two_neuron}%?? proper label
\end{figure}


\begin{figure}[h]
\includegraphics[width=\textwidth]{../img/two_neuron_wide.png} %?? Remove!
\caption{{\bf Overrepresentation of two neuron connections not is not
    induced by anisotropy}
A-B: While general overrepresentation is found, no significant
difference to the rewired mode is identified C-E: Overrepresentation
as found by Perin et al. is distance-independent and is not affected
by anisotropy}
\label{fig_two_neuron}%?? proper label
\end{figure}


% What organization principle then causes the overrepresentation?
By construction anisotropic, rewired and distance-dependent networks
share the same distance-dependent connection probabilities
(\figref{fig:4_net_models}G-H). It is thus natural to assume that is
the distance-dependent connectivity in the networks that induces the
overabundance of reciprocal connections. Indeed, as connection
probabilities within a pair are identical for both directions of
connection, an overrepresentation of bidirectional connection is
necessarily induced in the network as a result of Jensen's inequality
\cite{Hoffmann2017}. However, is this effect sufficient to explain the
experimentally observed overabundance of reciprocal connections? For
this we considered the probability of a pair at a given inter-neuron
distance to be bidirectionally connected
(\figref{fig:two_neuron}C-D). While the overall connection probability
in tuned anisotropic networks matched the data from
\textcite{Perin2011}, the distance-dependent probability for
reciprocally connected pairs was much lower than in the experimental
results. Taken together the results suggest that in cortical circuits
further connection principles that go beyond anisotropy in
connectivity drive. These could either Clopath be or other
non-symmetric Hoffmann.


% we turn to
% distance-dependent connectivity as a candidate .  Indeed, Because of
% its distance-dependency, the connection probability within a given
% pair is identical for both directions of conection in different
% network types (\figref{fig:4_net_models}G-H). However, is the effect
% of distance-dependence sufficient to explain the experimentally
% observed overabundance of reciprocal connections?


% The distance-dependent connectivity in the different network types
% (\figref{fig:4_net_models}G-H) is identical within  pairs and thus necessarily
% induces an overrepresentation of reciprocal connections in the network
% \cite{Hoffmann2017}. However, is the effect of distance-dependence
% sufficient to explain the experimentally observed overabundance of
% reciprocal connections?

% We found that rather than anisotropy, it reallz .

% Thus a further effect, likely in symmetric \cite{Hoffmann2017}.


% hof neurons appear in cortical circuits more
% often than would expect from a network that is connected randomly
% \cite{Markram1997, Song2005, Perin2011}.



