\clearpage
\pagebreak
\newpage
\subsection*{Three neuron motifs}

\begin{figure}[h]
%\includegraphics[width=0.99\textwidth]{../img/song_motifs.png} %??
%%Remove!

%\newgeometry{left=0.75in,right=0.75in} % ?? s
\centering
% \advance\leftskip-2.25in %??
% \includegraphics[width=6.2in]{../img/song_motif_single_img_colors.png}

\includegraphics[width=\textwidth]{../img/song_motif_single_img_colors.png}

%\includegraphics[width=6.8in]{/users/hoffmann/Downloads/9f9452a2_plos.pdf}
% ?? %Remove!
%\restoregeometry %??
\caption{{\bf Over- and underrepresentation of three-neuron motifs due
  to anisotropy}
  Counting the occurrences of three-neuron motifs in anisotropic,
  rewired and distance-dependent networks, we find that }
\label{fig_three-neuron}%?? proper label
\end{figure}


To expose the effect of anisotropy in connectivity on the statistics
of the occurrence of motifs composed of three neurons, in each of the
three graph types (anisotropic, rewired, distance-dependent) a sample
of 3 graphs was taken and in each of those graphs the motifs of 300000
random triplets of neurons were registered. % song_alt_dist_N3_300k 
The observed occurrence of three-neuron motifs distinctly differed
from what one might expect from relative frequencies of connections in
neuron pairs: Assuming independence, the expected relative frequency
of the 16 different motifs were calculated from probabilities $p_u,
p_s$ and $p_r$ from above %?? are they really mentioned abov?.  
\ref{SI_song_motif_comp}%??
Displaying the relative counts (counts extracted from graphs divided
by counts expected from neuron pair statistics) in
Fig.~\ref{fig_three-neuron}, we find that anisotropy has significant
impact of distribution of motifs. 

The strongest deviations were
observed in motifs 8,10,12,14,15 and 16 (overrepresented) and motif 9 ??
(underrepresented). Rewired and distance-dependent networks show a
deviation from the expectation as well, however, the effect is much
stronger in anisotropic networks

Interestingly, this effect is much stronger than in
distance-dependent, where one might expect significant over- and
underrepresentations as well. Finding two connected pairs in a triplet
indicates a spatial closeness, that can be transferred to third pair
as well - thus one should find a higher connection chance. We find
that in comparable distance-dependent networks, only


